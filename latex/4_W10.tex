\chapter{W10}
\section{第一題}
What is zmqRemoteAPI, and how does it relate to CoppeliaSim?

答

1.zmqRemoteAPI是 ZeroMQ 的遠程 API 通訊協議,用於在不同的程序之間進行通訊和數據傳輸。

2.可使CoppeliaSim進行遠端連接操控。


\section{第二題}
How do you establish a connection between a Python script and CoppeliaSim using zmqRemoteAPI?

答

python先下載zmq子模組,利用port:23000連接


\section{第三題}
What are some common use cases for zmqRemoteAPI in CoppeliaSim?

答

在 CoppeliaSim 中,zmqRemoteAPI 可以用於以下一些常見的應用場景:

控制機器人:zmqRemoteAPI 可以用於控制機器人在 CoppeliaSim中的運動和行為,包括設置關節位置、速度和力矩等參數,控制機器人的運動軌跡和姿態,以及獲取機器人的感測器數據和影像信息等。

編寫自動化測試:zmqRemoteAPI 可以幫助使用者編寫自動化測試脚本,測試機器人和其他物體的運動和行為,並驗證機器人的控制算法和程序的正確性。

設計自主導航系統:zmqRemoteAPI 可以用於設計自主導航系統,通過控制機器人的運動和行為來實現自主導航,並在 CoppeliaSim 中進行仿真測試。

進行物體檢測和跟蹤:zmqRemoteAPI 可以用於設計物體檢測和跟蹤系統,通過獲取 CoppeliaSim 中的影像數據和感測器數據來實現物體檢測和跟蹤功能。

zmqRemoteAPI 可以幫助使用者更加靈活方便控制 CoppeliaSim 中的機器人和物體。


\section{第四題}
What are the advantages and disadvantages of using zmqRemoteAPI compared to other methods of communication between Python and CoppeliaSim?

答

優點:

快速和高效:zmqRemoteAPI使用ZeroMQ消息庫,以其快速和高效的消息傳遞能力而聞名。
易於使用:zmqRemoteAPI是一個簡單易用的API,提供了一系列函數,可從Python腳本中控制模擬。
跨語言支持:zmqRemoteAPI是一種跨語言協議,因此您可以使用任何支持ZeroMQ的編程語言。
支持多個連接:zmqRemoteAPI支持多個連接,因此您可以將多個客戶端連接到單個CoppeliaSim實例。
缺點:

功能受限:儘管zmqRemoteAPI提供了一系列函數來控制CoppeliaSim,但與其他通信方法(如ROS或Python的Coppeliasim庫)相比,其功能受限。
上手難度高:zmqRemoteAPI需要一些ZeroMQ和socket編程的知識,這對於新手用戶來說可能不太容易使用。
可能出現錯誤:如果通信未正確配置,zmqRemoteAPI容易出現錯誤,這可能會導致消息丟失或模擬停滯等問題。
彈性較小:與其他通信方法相比,zmqRemoteAPI的自定義彈性較小,因為它依賴於預定義的一組函數。

\section{第五題}
Can you give an example of a project or task that you could complete using zmqRemoteAPI in CoppeliaSim?

答

1.開啟CoppeliaSim

2.選取場景

3.尋找主機IP位置

4.更改城市中的連結為至於上步驟之主機位置

5.開始連結


\section{小組工作分配}
41023219: 

設計零件導入場景。

41023221: 

討論查詢程式。

41023222: 

場景建設。

41023228: 

討論與設計程式,製作亂數。
\section{2b網站順序亂數}
\begin{lstlisting}[language=Python, frame=single, numbers=left, captionpos=b, basicstyle=\ttfamily\small, showstringspaces=false, breaklines=true, tabsize=4, xleftmargin=15pt]
from browser import html, document
import random
bcd_tem = "https://mdecd2023.github.io/2b2-pj2bg"
bgithub = "https://github.com/mdecd2023/2b2-pj2bg"
brython_div = document["brython_div1"]
#  亂數範圍從1到16
grp = []
for i in range(1, 17):
    grp.append(i)
random.shuffle(grp)
for i in grp:
    url = bcd_tem + str(i)
    github = bgithub + str(i)
    brython_div <= html.A("pj2bg"+str(i), href=url)
    brython_div <= " ("
    brython_div <= html.A("repo", href=github)
    brython_div <= ")"
    brython_div <= html.BR()
\end{lstlisting}
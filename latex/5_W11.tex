\chapter{W11}


\section{41023219}

負責部分:單機計分\\過程:修改感測器的程式檔,分別定義四台機器人為bubbleRob1、2、3、4,然後修改細項,讓感測器能分辨球與機器人。\\自評:這次的分組讓我學會了如何解決上傳合併的衝突,然後在修改計分系統的時候遇到了一些問題,很感謝組長的協助。\\自評分數:60


\section{41023221}

負責部分:完整版\\過程:先編輯計分表把它區分成四大部分,分別為紅方得分十位數、紅方得分個位數、綠方十位數以及綠方個位數,再把四大部分的每個零件編號,以利於後面進行變色,利用個別零件的變色,來達到顯示分數,最後把它合併進場景再進行程式代碼除錯(因為是分開寫完再合併的會有衝突),主要錯誤在於感應器,把名稱對上就行。\\自評:遇到需多困境與難題,所幸在與同學的討論中取得成功。\\自評分數:64

\section{41023222}

負責部分:足球場景\\自評:還在努力跟上組別的進度。\\自評分數:60分


\section{41023228}

負責部分:控制系統\\過程:做出四隻機器人,更改按鍵,在場景內加入code讓單機時有計分系統,在場景中加入七段顯示器,得分時變色顯示分數,讓玩家在瀏覽器中可以看到分數。\\自評:pj2中負責做出PDF報告,w10做出亂數、w11主要做控制系統,輔助組員做單機顯示分數及瀏覽器也可以看到分數的實體顯示,pj2結束後持續幫助組員及其他組別解決問題。\\自評分數:65分